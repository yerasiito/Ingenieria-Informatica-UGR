\documentclass[12pt, spanish]{article}
\usepackage[spanish]{babel}
\selectlanguage{spanish}
\usepackage{natbib}
\usepackage{url}
\usepackage[utf8x]{inputenc}
\usepackage{graphicx}
\graphicspath{{images/}}
\usepackage{array}
\usepackage{longtable}
\usepackage{parskip}
\usepackage{fancyhdr}
\usepackage{vmargin}
\usepackage{listings}
\usepackage{adjustbox}
\usepackage[shortlabels]{enumitem}

\setlist[enumerate]{
labelsep=8pt,
labelindent=0.3\parindent,
itemindent=0pt,
leftmargin=*,
before=\setlength{\listparindent}{-\leftmargin},
}

\usepackage[default]{sourcesanspro}

\setmarginsrb{2 cm}{1 cm}{2 cm}{2 cm}{1 cm}{1.5 cm}{1 cm}{1.5 cm}

\title{Fundamentos de Ingeniería del Software:\\
Práctica 1  \hspace{0.05cm} }
\date{}
\author{
\begin{center}
Martina Álvarez Lorenzo  \\
Pablo Ariza García  \\
David Gutiérrez Pérez  \\
Yeray López Ramírez \\
\end{center}
}

\renewcommand*\contentsname{hola}

\makeatletter
\let\thetitle\@title
\let\theauthor\@author
\let\thedate\@date
\makeatother

\pagestyle{fancy}
\fancyhf{}
\rhead{}
\chead{\thedate}
\lhead{\thetitle}
\cfoot{\thepage}

\begin{document}
%%%%%%%%%%%%%%%%%%%%%%%%%%%%%%%%%%%%%%%%%%%%%%%%%%%%%%%%%%%%%%%%%%%%%%%%%%%%%%%%%%%%%%%%%

\begin{titlepage}
  \centering
  \vspace*{0.5 cm}
  \includegraphics[scale = 0.50]{ugr.png}\\[1.0 cm]
  %\textsc{\LARGE Universidad de Granada}\\[2.0 cm]   
  \textsc{\large 2ºD}\\[0.5 cm]
  \textsc{\large Grado en Ingeniería Informática}\\[0.5 cm]              
  \rule{\linewidth}{0.2 mm} \\[0.4 cm]
  { \huge \bfseries \thetitle}\\
  \rule{\linewidth}{0.2 mm} \\[1.5 cm]
  
      
  \theauthor

  \vfill
  
\end{titlepage}

\newpage

%%%%%%%%%%%%%%%%%%%%%%%%%%%%%%%%%%%%%%%%%%%%%%%%%%%%%%%%%%%%%%%%%%%%%%%%%%%%%%%%%%%%%%%%%


\tableofcontents
\pagebreak

%%%%%%%%%%%%%%%%%%%%%%%%%%%%%%%%%%%%%%%%%%%%%%%%%%%%%%%%%%%%%%%%%%%%%%%%%%%%%%%%%%%%%%%%%

\section{Objetivos}

El sistema estará encargado de realizar la gestión de un centro médico privado, cuya tarea es proporcionar distintos servicios médicos a sus socios, ya sea una atención básica del paciente, servicios especializados o urgencias.

El sistema deberá gestionar los pacientes, profesionales de la salud e instalaciones existentes en el centro médico privado. Deberá almacenar la información relativa a los pacientes atendidos, el horario de los profesionales, citas, pruebas médicas, instalaciones y tratamientos disponibles.

De manera concisa, los objetivos a realizar son:
\begin{enumerate}[start=1,label={\bfseries OBJ-\arabic*.}, leftmargin=2cm,listparindent=-\leftmargin]
  \item Gestión de la Información: Se deben tratar los datos de los pacientes internados, personal médico y material e instalaciones disponibles.
  \item Control de citas: Administrar y asignar citas entre pacientes y profesionales.
  
  \item Asignación de pruebas médicas a pacientes en las instalaciones disponibles.
  
  \item Administración de material e instalaciones médicas disponibles o requeridas.
  
  \item Gestión administrativa del personal.
  
\end{enumerate}


\section{Descripción de los implicados y usuarios finales}

\textbf{Entorno de usuario}

Los usuarios directos del sistema que vamos a desarrollar son dos, el médico y el paciente. Su nivel cultural es bajo-medio. Ambos tienen experiencia previa con aplicaciones informáticas pero no se desenvuelven con soltura, no saben reaccionar en caso de que el sistema falle. \\

%Tabla del personal
\begin{longtable}{|p{.2\linewidth}|p{.25\linewidth}|c|p{.25\linewidth}|}
  \caption{Jerarquía del personal} \\
  \hline
  \centering{\textbf{Nombre}}  & \centering{\textbf{Descripción}} & \centering{\textbf{Tipo}} & \centering{\textbf{Responsabilidad}}
  \endhead
  \hline
  \centering Médico & Representa el principal trabajador del sistema sanitario & Usuario Producto & Responsabilidad máxima en el trato de la salud del paciente.\\
  \hline
  \centering Enfermero & Representa a un trabajador del sistema sanitario. Se encuentra bajo el mando del médico supervisor del paciente. & Usuario producto & Responsabilidad secundaria en el trato del paciente.\\
  \hline
  \centering Camillero & Representa un auxiliar de enfermería & Usuario producto & Se encarga del mantenimiento de las camas de los pacientes. \\
  \hline
  \centering Paciente & Representa el usuario de los servicios & Usuario sistema & Seguir las indicaciones del personal sanitario. \\
  \hline
  \centering Celador & Representa el personal encargado de realizar distintas acciones dentro del hospital, encargándose de que todo esté en orden & Usuario producto & Se encarga de trasladar a pacientes e instrumental a diferentes lugares del hospital. \\
  \hline
  \centering\arraybackslash Atención al cliente & Representa al encargado de atender a clientes en recepción, ya sea en persona o por teléfono. & Usuario producto & Atender las preguntas, dudas, solicitudes del cliente. Se encargan de que los pacientes e instrumental se encuentren en el lugar adecuado en el momento adecuado. \\
  \hline
  \centering\arraybackslash Director del centro & Es el máximo responsable del centro. Es el principal representante de la institución. & Usuario producto & Llevar a cabo las principales tareas de gestión del centro. \\
  \hline
  \centering\arraybackslash Proveedor de material médico & Representa un proveedor & Usuario sistema & Suministrar material médico al hospital \\
  \hline
\end{longtable}

\renewcommand{\arraystretch}{1.5}
%Resto de tablas
\begin{longtable}{|c|p{.65\linewidth}|}
   \caption{Médico} \\
   \hline
   Representante & José Luis Zamorano \\
   \hline
   Descripción & Médico \\
   \hline
   Tipo & Experto \\
   \hline
   Responsabilidades & Revisión, indagación, trato y mejora de la salud del paciente. Dirección de los enfermeros asignados a los pacientes que trate. Investigación, estudio y/o aplicación de nuevos o mejores tratamientos. \\
   \hline
   Criterios de éxito & Baja tasa de mortalidad. Tratamientos innovadores. Tasa de curación o mejora de la salud del paciente. \\
   \hline
   Implicación & Es el principal encargado de la salud y vida del paciente. \\
   \hline
   Comentarios/Cuestiones & Está familiarizado con los principales campos de la medicina y especializado en uno de ellos. \\
   \hline
\end{longtable}


\begin{longtable}{|c|p{.65\linewidth}|}
   \caption{Enfermero} \\
   \hline
   Representante & Florence Nightingale \\
   \hline
   Descripción & Enfermero \\
   \hline
   Tipo & Representa a un trabajador del sistema sanitario. Se encuentra bajo el mando del médico supervisor del paciente. \\
   \hline
   Responsabilidades & Atención óptima, atención oportuna y contínua, atención cuidadosa, acatamiento de instrucciones del médico tratante, consentimiento escrito previo explicación para procedimientos riesgosos. \\
   \hline
   Criterios de éxito & Hay éxito si consiguen cumplir cada una de sus tareas de forma efectiva. \\
   \hline
   Implicación & Es el encargado de atender a los pacientes en determinados escenarios por ejemplo, un paciente con una herida, el enfermero o enfermera se encargará de limpiar la herida con suero y aplicarle en la zona una gasa para proteger. \\
   \hline
   Comentarios/Cuestiones & Está en constante contacto con los pacientes y sus familiares. \\
   \hline
\end{longtable}


\begin{longtable}{|c|p{.65\linewidth}|}
   \caption{Camillero} \\
   \hline
   Representante & Juán Gómez\\
   \hline
   Descripción & Encargado \\
   \hline
   Tipo & Camillero \\
   \hline
   Responsabilidades & Trasladar pacientes con diferentes patologías, tanto para la silla de ruedas como en camilla (llevar a una persona desde una habitación hasta el quirófano), poniendo en práctica las normas de seguridad. \\
   \hline
   Criterios de éxito & Hay éxito si cumplen cada una de sus tareas en un tiempo determinado. \\
   \hline
   Implicación & Es el responsable del mantenimiento y asignación de camas a los distintos pacientes del sistema. \\
   \hline
   Comentarios/Cuestiones & \\
   \hline
\end{longtable}


\begin{longtable}{|c|p{.65\linewidth}|}
   \caption{Paciente} \\
   \hline
   Representante & Alberto Alcántara\\
   \hline
   Descripción & Paciente \\
   \hline
   Tipo & No utiliza el sistema de forma directa sino que desencadena que otros usuarios hagan uso del mismo, además será un usuario casual. \\
   \hline
   Responsabilidades & Seguir las indicaciones del personal médico.  \\
   \hline
   Criterios de éxito & Que pueda consultar y pedir cita con facilidad. Además de ofrecer una atención personalizada. \\
   \hline
   Implicación & Utilizará el sistema cada vez que necesite atención médica o revisión. \\
   \hline
   Comentarios/Cuestiones & \\
   \hline
\end{longtable}


\begin{longtable}{|c|p{.65\linewidth}|}
   \caption{Celador} \\
   \hline
   Representante & Ana Maria Muñoz\\
   \hline
   Descripción & Asistente médico \\
   \hline
   Tipo & Representa al personal del equipo médico.  \\
   \hline
   Responsabilidades & Se encarga de trasladar a pacientes e instrumental a diferentes lugares del hospital.  \\
   \hline
   Criterios de éxito & Trato adecuado del paciente hasta ser tratado por enfermeros o médicos. \\
   \hline
   Implicación & Realiza el primer contacto del paciente al acceder al servicio sanitario. \\
   \hline
   Comentarios/Cuestiones & \\
   \hline
\end{longtable}


\begin{longtable}{|c|p{.65\linewidth}|}
   \caption{Atención al Cliente/Paciente} \\
   \hline
   Representante & Rosa García\\
   \hline
   Descripción & Atender al cliente con problemas técnicos o administrativos. \\
   \hline
   Tipo & Representa al encargado de atender a los pacientes en recepción, ya sea en persona o por teléfono.  \\
   \hline
   Responsabilidades & Gestión de las citas y los documentos requeridos por los pacientes y los médicos. Además de resolver las dudas que puedan surgir a los pacientes. \\
   \hline
   Criterios de éxito & Resolver el problema, duda o solicitud que tiene el cliente o usuario. \\
   \hline
   Implicación & Encargada de atender a pacientes, de forma de consulta (acerca de citas como por ejemplo, planta de la consulta médica y hora). \\
   \hline
   Comentarios/Cuestiones & \\
   \hline
\end{longtable}

\begin{longtable}{|c|p{.65\linewidth}|}
   \caption{Director del Centro} \\
   \hline
   Representante & Jose María Camara\\
   \hline
   Descripción & Máximo representante del centro.\\
   \hline
   Tipo & Administrador.  \\
   \hline
   Responsabilidades & Gestión de las principales áreas médicas y del presupuesto. Además, es la cara visible del centro. \\
   \hline
   Criterios de éxito & Una buena gestión presupuestaria.

   Conseguir en general un trato correcto al paciente.

   Disponer de un personal médico de excelencia.

   Tener una baja tasa de mortalidad.

   Disponer de tratamientos innovadores. \\
   \hline
   Implicación & Indirecta, como supervisor del sistema y del personal sanitario \\
   \hline
   Comentarios/Cuestiones & \\
   \hline
\end{longtable}


\begin{longtable}{|c|p{.65\linewidth}|}
   \caption{Proveedor del material médico} \\
   \hline
   Representante & Raúl Pérez\\
   \hline
   Descripción & Proveedor. \\
   \hline
   Tipo & No utiliza el sistema de forma directa sino que desencadena que otros usuarios hagan uso del mismo.  \\
   \hline
   Responsabilidades & Suministrar material médico al hospital. \\
   \hline
   Criterios de éxito & Que el sistema le permita realizar sus actividades de la forma más sencilla posible. \\
   \hline
   Implicación & Indirecta, como suministrador de información para el sistema \\
   \hline
   Comentarios/Cuestiones & \\
   \hline
\end{longtable}


\begin{longtable}{|p{.15\linewidth}|c|p{.2\linewidth}|p{.2\linewidth}|p{.25\linewidth}|}
  \caption{Necesidades de los implicados} \\
  \hline
  \centering{\textbf{Necesidad}}  & \centering{\textbf{Prioridad}} & \centering{\textbf{Problema}} &  \centering{\textbf{Solución actual}} & \centering{\textbf{Solución propuesta}}
  \endhead
  \hline
  \centering Operación & Alta & El cirujano no sabe que parte del paciente debe operar & El cirujano pregunta al paciente & Registrar en el sistema la operación a realizar\\
  \hline
  \centering Cita médica & Media & Cómo sabe el médico a que hora le corresponde a cada paciente & Los horarios de las citas no se respetan. El que antes llegue, se atiende & Se lleva un histórico de las citas del día \\
  \hline
  \centering Recursos Médicos & Alta & No se sabe de qué material se dispone, sobran y faltan materiales médicos & Se hace una lista a papel de las herramientas y materiales que faltan & Se crea un catálogo con todos los materiales de los que dispone el centro en tiempo real \\
  \hline
  \centering Jornadas de vacunación & Media & No hay forma de contactar con los pacientes que deben ser vacunados & Se distribuyen carteles por el centro y por zonas estratégicas de la ciudad/pueblo & Se avisa por e-mail o sms a los pacientes a los que se destina la vacuna \\
  \hline
  \centering Recordatorios de cita & Baja & Los pacientes olvidan sus citas con frecuencia & Se manda un correo al paciente & El sistema le mandará una notificación vía sms o app \\
  \hline
\end{longtable}

\clearpage
\section{Requisitos funcionales}

\begin{enumerate}[start = 1, label={\bfseries RF-\arabic*.}]
  \item Gestión de la Información \\
  El sistema deberá gestionar los datos de los pacientes internados, personal médico y material e instalaciones disponibles.
  \begin{enumerate}[label*={\bfseries \arabic*.}]
    \item Se llevará un control de los datos de cada uno de los pacientes incluyendo su estado e historial médico.
    \item El sistema nos permitirá llevar un control de las ofertas de los proveedores y de los pedidos que se realicen.
    \begin{enumerate}[label*={\bfseries \arabic*.}]
      \item Necesitamos mantener una lista con los distintos proveedores que contenga información sobre su contacto. Pudiendo actualizar, añadir y consultar esa lista en cualquier momento.
      \item Guardar los datos de un pedido realizado por un proveedor
      \item Añadir información sobre nuevos materiales sanitarios del centro médico.
    \end{enumerate}
    \item Ver el estado en el que se encuentran los pedidos
    \item Añadir la información de pacientes para solicitar citas a través de aplicaciones móviles.
  \end{enumerate}
  \item Control de citas: Administrar y asignar citas entre pacientes y profesionales.
  \begin{enumerate}[label*={\bfseries \arabic*.}]
    \item Añadir y consultar citas de un determinado paciente
    \item Obtener información sobre una cita en concreto o ver el informe de una cita pasada.
    \begin{enumerate}[label*={\bfseries \arabic*.}]
      \item  Ver la fecha de la cita
      \item  Ver informe de la cita
      \item  Modificar la fecha de una cita por incompatibilidad por parte del paciente
      \item  Pago de las citas
      \begin{enumerate}[label*={\bfseries \arabic*.}]
        \item Llevar un listado de las citas de cada paciente y la gestión de pago
      \end{enumerate}
    \end{enumerate}
  \end{enumerate}
\end{enumerate}

\clearpage
\section{Requisitos no funcionales}

\paragraph{Usabilidad}
Esta sección lista todos los requisitos relativos a la usabilidad del sistema.
\begin{enumerate}[start =1, label={\bfseries RN-\arabic*.}]
  \item Debe proporcionarse ayuda en línea con instrucciones paso a paso para guiçar al médico y/o paciente en la información que deba consultar.
  \item Se deben comunicar avisos y señales visuales y/o con sonidos, por ejemplo, para avisar de que acaba de entrar o salir un paciente al centro ya que el recepcionista no está continuamente en la sala de entrada.
  \item Se usarán asistentes que guíen en los procedimientos de vacunación o solicitud de cita a los pacientes mayores o con desconocimiento de la informática.
\end{enumerate}

\paragraph{Fiabilidad}
\begin{enumerate}[resume*]
  \item Mantener el acceso de los pacientes controlado asignando un usuario y contraseña para acceder a sus datos privados. 
  \item Para prever caídas del sistema se harán back-ups y se contratará una protección contra DDOS.
  \item Se empleará el protocolo HTTPS.
  \item Se mantendrán partes del sistema manual actual junto con el sistema propuesto, en caso de que el sistema fallara o se produjera un corte de conexión.
\end{enumerate}

\paragraph{Rendimiento}
\begin{enumerate}[resume*]
  \item Se usarán dos terminales distintas: una para el médico y otra para el paciente, mejorando así la velocidad del sistema y su seguridad.
  \item Las peticiones de cita se generarán automáticamente para acelerar el proceso de asignación e información del paciente y del médico
  \item El sistema proporcionará acceso rápido al calendario de citas y vacunación de la base de datos, no tardando más de 5 segundos. Se calcula que el sistema debe manejar un volumen de datos de 2.000 pacientes y 50 médicos.
\end{enumerate} 

\paragraph{Soporte}
\begin{enumerate}[resume*]
  \item Los datos que reciben los proveedores deben ser adaptados para que sean compatibles con la base de datos del centro.
\end{enumerate}

\paragraph{Restricciones de implementación}
\begin{enumerate}[resume*]
  \item Se debe usar el lenguaje C++ compatible con c++11 y cmake 2.8
\end{enumerate}

\paragraph{Restricciones de interfaz}
\begin{enumerate}[resume*]
  \item  El sistema debe interactuar y recoger datos de los pacientes vía Intranet, conectándose a la base de datos del centro.
\end{enumerate}

\paragraph{Requisitos físicos}
\begin{enumerate}[resume*]
  \item Los medicamentos de prescripción médica deben ser recetados mediante un documento físico firmado y sellado por el médico para que el paciente los recoja en la farmacia oportuna. Suelen incluir un QR o un código de barras con el medicamento prescrito.
\end{enumerate}

\section{Requisitos de información}

\begin{enumerate}[start =1, label={\bfseries RI-\arabic*.}]
  \item Pacientes internados. 
  \\ Información sobre el enfermo que se encuentra en tratamiento. 
  \\ \textbf{Contenido:} Datos personales, DNI, historial médico, diagnóstico, posible fecha de alta médica
  \\ \textbf{Requisitos relacionados:} RI-2
  \item Socios 
  \\ Información sobre el socio adscrito al servicio.
  \\ \textbf{Contenido:} Número de socio, datos personales, DNI, historial médico, pagos realizados y pendientes, fecha de alta como socio, nivel de seguro contratado.
  \\ \textbf{Requisitos relacionados:} RI-1
  \item Profesionales sanitarios 
  \\ Información sobre el profesional sanitario que realiza los servicios.
  \\ \textbf{Contenido:} Datos personales, DNI, número seguridad social, horarios, cargo y especialización.
  \\ \textbf{Requisitos relacionados:} RI-4
  \item Instalaciones 
  \\ Información sobre las instalaciones médicas disponibles en el centro y los procesos que se realizan en ellas 
  \\ \textbf{Contenido:} Nombre, localización (Planta y Habitación), pruebas diagnósticas asociadas y personal encargado.
  \\ \textbf{Requisitos relacionados:} RI-2, RI-5
  \item Pruebas diagnosticas 
  \\ Información sobre las diferentes pruebas que se encuentran disponibles y dónde se realizan.
  \\ \textbf{Contenido:} Nombre, descripción, instalación asociada y material requerido.
  \\ \textbf{Requisitos relacionados:} RI-4, RI-6
  \item Material médico 
  \\ Descripción del material médico necesario para los servicios y su disponibilidad. 
  \\ \textbf{Contenido:} Nombre, existencias
  \\ \textbf{Requisitos relacionados:} RI-5, RI-6, RI-7
  \item Proveedores  
  \\ Información sobre los proveedores que suministran los materiales y equipamiento necesarios.
  \\ \textbf{Contenido:} Nombre, información de contacto, historial de pedidos, material que suministran.
  \\ \textbf{Requisitos relacionados:} RI-6, RI-8
  \item Pedidos  
  \\ Pedidos de material o aparatos que se encuentran en reparto. 
  \\ \textbf{Contenido:} Contenido del pedido, localizador de paquete, contacto de mensajería, proveedor “solicitado”, fecha de realización del pedido, fecha de entrega del pedido.
  \\ \textbf{Requisitos relacionados:} RI-5, RI-6
\end{enumerate}


\vspace{5cm}

\bibliographystyle{plain}
\bibliography{biblist}


\end{document}
