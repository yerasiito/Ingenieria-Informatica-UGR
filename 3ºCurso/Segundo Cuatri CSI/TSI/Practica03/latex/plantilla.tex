%%%%%%%%%%%%%%%%%%%%%%%%%%%%%%%%%%%%%%%%%
% Short Sectioned Assignment LaTeX Template Version 1.0 (5/5/12)
% This template has been downloaded from: http://www.LaTeXTemplates.com
% Original author:  Frits Wenneker (http://www.howtotex.com)
% License: CC BY-NC-SA 3.0 (http://creativecommons.org/licenses/by-nc-sa/3.0/)
%%%%%%%%%%%%%%%%%%%%%%%%%%%%%%%%%%%%%%%%%

%----------------------------------------------------------------------------------------
%	PACKAGES AND OTHER DOCUMENT CONFIGURATIONS
%----------------------------------------------------------------------------------------

\documentclass[paper=a4, fontsize=11pt]{scrartcl} % A4 paper and 11pt font size
% ---- Entrada y salida de texto -----

\usepackage[T1]{fontenc} % Use 8-bit encoding that has 256 glyphs
\usepackage[utf8]{inputenc}
%\usepackage[default]{sourcesanspro}
\usepackage{fourier} % Use the Adobe Utopia font for the document - comment this line to return to the LaTeX default
% ---- Idioma --------

\usepackage[spanish, es-tabla]{babel} % Selecciona el español para palabras introducidas automáticamente, p.ej. "septiembre" en la fecha y especifica que se use la palabra Tabla en vez de Cuadro

% ---- Code insertion ----
\usepackage{minted}
\usepackage[breakable]{tcolorbox} %Genera boxes, lo usamos para el codigo
\BeforeBeginEnvironment{minted}{\begin{tcolorbox}[breakable]}%
	\AfterEndEnvironment{minted}{\end{tcolorbox}}%
\BeforeBeginEnvironment{inputminted}{\begin{tcolorbox}[breakable]}%
	\AfterEndEnvironment{inputminted}{\end{tcolorbox}}%

\usepackage{xpatch} %permite abrir environments al ejecutar ciertos comandos

\xpretocmd{\inputminted}{\begin{tcolorbox}}{}{}%
	\xapptocmd{\inputminted}{\end{tcolorbox}}{}{}%

\setminted{
fontsize=\small,
breaklines,
breaksymbolleft=
}
% ---- Otros paquetes ----
\usepackage{dirtree}
\usepackage{vmargin}

\usepackage{url} % ,href} %para incluir URLs e hipervínculos dentro del texto (aunque hay que instalar href)
\usepackage{hyperref} % ,href} %para incluir URLs e hipervínculos dentro del texto (aunque hay que instalar href)
\hypersetup{
	colorlinks=true,
	linkcolor=blue,
	filecolor=magenta,      
	urlcolor=blue,
}

\usepackage{xcolor}
\definecolor{light-gray}{gray}{0.95}
\definecolor{alizarin}{rgb}{0.82, 0.1, 0.26}
\definecolor{nyellow}{rgb}{0.91, 0.656, 0.0}
%\definecolor{indigo}{rgb}{0.29, 0.0, 0.51}
%	\textcolor{red}{p}
%	\textcolor{orange}{r}
%	\textcolor{yellow}{a}
%	\textcolor{green}{c}
%	\textcolor{blue}{t}
%	\textcolor{indigo}{i}
%	\textcolor{violet}{c}
%	\textcolor{red}{a}
%	\textcolor{orange}{s}
%	\textcolor{yellow}{,}
%	\textcolor{green}{I}
%	\textcolor{blue}{S}
%	\textcolor{indigo}{E}

\usepackage{amsmath,amsfonts,amsthm} % Math packages
%\usepackage{graphics,graphicx, floatrow} %para incluir imágenes y notas en las imágenes
\usepackage{graphics,graphicx, float} %para incluir imágenes y colocarlas
\usepackage[export]{adjustbox}
\graphicspath{{imagenes/}}

% Para hacer tablas comlejas
%\usepackage{multirow}
%\usepackage{threeparttable}

%\usepackage{sectsty} % Allows customizing section commands
%\allsectionsfont{\centering \normalfont\scshape} % Make all sections centered, the default font and small caps

%Remove warnings
\usepackage{silence}
\WarningFilter{scrartcl}{Usage of package `fancyhdr'}

\usepackage{fancyhdr} % Custom headers and footers
\pagestyle{fancyplain} % Makes all pages in the document conform to the custom headers and footers
\fancyhead{} % No page header - if you want one, create it in the same way as the footers below
\fancyfoot[L]{} % Empty left footer
\fancyfoot[C]{} % Empty center footer
\fancyfoot[R]{\thepage} % Page numbering for right footer
\renewcommand{\headrulewidth}{0pt} % Remove header underlines
\renewcommand{\footrulewidth}{0pt} % Remove footer underlines
\setlength{\headheight}{13.6pt} % Customize the height of the header

\numberwithin{equation}{section} % Number equations within sections (i.e. 1.1, 1.2, 2.1, 2.2 instead of 1, 2, 3, 4)
\numberwithin{figure}{section} % Number figures within sections (i.e. 1.1, 1.2, 2.1, 2.2 instead of 1, 2, 3, 4)
\numberwithin{table}{section} % Number tables within sections (i.e. 1.1, 1.2, 2.1, 2.2 instead of 1, 2, 3, 4)

\setlength\parindent{0pt} % Removes all indentation from paragraphs - comment this line for an assignment with lots of text

\newcommand{\horrule}[1]{\rule{\linewidth}{#1}} % Create horizontal rule command with 1 argument of height

\setmarginsrb{2 cm}{1 cm}{2 cm}{2 cm}{1 cm}{1.5 cm}{1 cm}{1.5 cm} %Aumenta los márgenes

%----------------------------------------------------------------------------------------
%	TÍTULO Y DATOS DEL ALUMNO
%----------------------------------------------------------------------------------------

\title{
	Práctica 3 \\\vspace{1cm}
	Técnicas de Sistemas Inteligentes \vspace{1cm} \\
	Planificación Clásica (PDDL) \vspace{1cm} \\
 }   

\author{Yeray López Ramírez	}                             

\renewcommand*\contentsname{hola}

\makeatletter
\let\thetitle\@title
\let\theauthor\@author
\let\thedate\@date
\makeatother

%----------------------------------------------------------------------------------------
% DOCUMENTO
%----------------------------------------------------------------------------------------

\begin{document}
\begin{titlepage}
	\centering
	%\textsc{\LARGE Universidad de Granada}\\[2.0 cm]    
	\includegraphics[scale = 0.6]{ugr.png}\\[1.0 cm]
	\rule{\linewidth}{0.2 mm} \\[0.4 cm]
	{ \huge \bfseries \thetitle}\\
	\rule{\linewidth}{0.2 mm} \\[1.5 cm]
	
	\begin{minipage}{0.5\textwidth}
		\begin{flushleft} \large
			\theauthor 
			123456789-Z \\
			\href{mailto:fulanito@correo.ugr.es}{fulanito@correo.ugr.es}
		\end{flushleft}
	\end{minipage}~
	\begin{minipage}{0.5\textwidth}
		\begin{flushright} \large
			Curso: 2022-23 \\
			Grupo A, 15:30 - 17:30 (Martes)                   
		\end{flushright}
	\end{minipage}\\[1 cm]
	
	{\small \thedate}\\[1 cm]
	
	\vfill
	
\end{titlepage}


\newpage %inserta un salto de página
\newcommand{\code}[1]{\colorbox{light-gray}{\textcolor{alizarin}{\texttt{#1}}}}
\newcommand{\high}[1]{\colorbox{light-gray}{\textcolor{nyellow}{\texttt{#1}}}}

\tableofcontents % para generar el índice de contenidos

\listoftables

\newpage

%----------------------------------------------------------------------------------------
%	Cuestión 1
%----------------------------------------------------------------------------------------

\section{Tabla de resultados}
\subsection{Ejercicio 1}
\noindent\fbox{%
	\parbox{0.3\textwidth}{\textbf{Objetivo:}
producir minerales.
	}%
}

\input{ejer1.tex}
Número de acciones: 4 de navegar y 1 de asignar.\\
Tiempo: El plan resultante es sencillo de calcular por lo que es instantáneo.

\subsection{Ejercicio 2}
\noindent\fbox{%
	\parbox{0.3\textwidth}{\textbf{Objetivo:}
		producir gas vespeno.
	}%
}
% Please add the following required packages to your document preamble:
% \usepackage{graphicx}
% \usepackage[table,xcdraw]{xcolor}
% If you use beamer only pass "xcolor=table" option, i.e. \documentclass[xcolor=table]{beamer}
\begin{table}[H]
	\centering
	\caption{Resultados del ejercicio 2}
	\label{tab:ejercicio2}
	\resizebox{0.4\textwidth}{!}{%
		\begin{tabular}{|c|c|c|}
			\hline
			\rowcolor[HTML]{EFEFEF} 
			\textbf{ejercicio}                 & \textbf{Nºacciones} & \textbf{Tiempo(s)} \\ \hline
			\cellcolor[HTML]{DAE8FC}\textbf{2} & 13                  & 0.00               \\ \hline
		\end{tabular}%
	}
\end{table}

Número de acciones: 10 de navegar, 2 de asignar y 1 de construir.\\
Tiempo: El plan resultante es sencillo de calcular por lo que es instantáneo de nuevo.

\subsection{Ejercicio 3}
\noindent\fbox{%
	\parbox{0.57\textwidth}{\textbf{Objetivo:}
		construir barracones en una localización concreta.
	}%
}
\input{ejer3.tex}

Número de acciones: 17 de navegar, 2 de asignar y 2 de construir.\\
Tiempo: El plan resultante empieza a tener cierta complejidad. Sigue siendo milisegundos.

\subsection{Ejercicio 4}
\noindent\fbox{%
	\parbox{0.8\textwidth}{\textbf{Objetivo:}
		reclutar 2 marines y 1 soldado. Enviarlos a localizaciones concretas.
	}%
}

% Please add the following required packages to your document preamble:
% \usepackage{graphicx}
% \usepackage[table,xcdraw]{xcolor}
% If you use beamer only pass "xcolor=table" option, i.e. \documentclass[xcolor=table]{beamer}
\begin{table}[H]
	\centering
	\caption{Resultados del ejercicio 4}
	\label{tab:ejercicio4}
	\resizebox{0.4\textwidth}{!}{%
		\begin{tabular}{|c|c|c|}
			\hline
			\rowcolor[HTML]{EFEFEF} 
			\textbf{ejercicio}                 & \textbf{Nºacciones} & \textbf{Tiempo(s)} \\ \hline
			\cellcolor[HTML]{DAE8FC}\textbf{4} & 30                  & 0.01               \\ \hline
		\end{tabular}%
	}
\end{table}

Número de acciones: 21 de navegar, 5 de reclutar (2 VCEs + objetivo), 2 de construir y 2 de asignar.\\
Tiempo: El plan resultante vuelve a ser casi instantáneo tras añadir reglas redundantes.

\subsection{Ejercicio 5}
\noindent\fbox{%
	\parbox{0.8\textwidth}{\textbf{Objetivo:}
		reclutar 2 marines y 1 soldado. Soldado requiere investigar spartan.
	}%
}

% Please add the following required packages to your document preamble:
% \usepackage{graphicx}
% \usepackage[table,xcdraw]{xcolor}
% If you use beamer only pass "xcolor=table" option, i.e. \documentclass[xcolor=table]{beamer}
\begin{table}[H]
	\centering
	\caption{Resultados del ejercicio 5}
	\label{tab:ejercicio5}
	\resizebox{0.4\textwidth}{!}{%
		\begin{tabular}{|c|c|c|}
			\hline
			\rowcolor[HTML]{EFEFEF} 
			\textbf{ejercicio}                 & \textbf{Nºacciones} & \textbf{Tiempo(s)} \\ \hline
			\cellcolor[HTML]{DAE8FC}\textbf{5} & 32                  & 0.04               \\ \hline
		\end{tabular}%
	}
\end{table}

Nº de acciones: 21 de navegar, 5 de reclutar, 3 de construir, 2 de asignar y 1 de investigar. No es óptimo.\\
Tiempo: El plan resultante vuelve a ser casi instantáneo tras añadir reglas redundantes.

\subsection{Ejercicio 6}
\noindent\fbox{%
	\parbox{0.6\textwidth}{\textbf{Objetivo:}
		mismo que ejercicio 5. Restringir número de acciones.
	}%
}

% Please add the following required packages to your document preamble:
% \usepackage{graphicx}
% \usepackage[table,xcdraw]{xcolor}
% If you use beamer only pass "xcolor=table" option, i.e. \documentclass[xcolor=table]{beamer}
\begin{table}[H]
	\centering
	\caption{Resultados del ejercicio 6}
	\label{tab:ejercicio6}
	\resizebox{0.5\textwidth}{!}{%
		\begin{tabular}{|c|c|c|}
			\hline
			\rowcolor[HTML]{EFEFEF} 
			\textbf{ejercicio}                                     & \textbf{Nºacciones} & \textbf{Tiempo(s)} \\ \hline
			\cellcolor[HTML]{DAE8FC}\textbf{6 (sin restringir)}    & 32                  & 0.07               \\ \hline
			\cellcolor[HTML]{DAE8FC}\textbf{6 (restringido <= 28)} & 28                  & 1.23               \\ \hline
			\cellcolor[HTML]{DAE8FC}\textbf{6 (proven unsolvable)} & {\color[HTML]{FE0000} unsolvable} & 5.61 \\ \hline
		\end{tabular}%
	}
\end{table}

Nº de acciones: 
\begin{itemize}
	\item Sin restringir: 21 de navegar, 5 de reclutar, 3 de construir, 2 de asignar y 1 de investigar. No es óptimo.
	\item Restringido: 17 de navegar, 5 de reclutar, 3 de construir, 2 de asignar y 1 de investigar. Es óptimo.
\end{itemize}

Tiempo: esta vez al restringir el tiempo aumenta por encima del segundo. Tarda 5 veces más en probar que no se puede realizar en menos acciones.

\subsection{Ejercicio 7}
\noindent\fbox{%
	\parbox{0.6\textwidth}{\textbf{Objetivo:}
		mismo que ejercicio 4. Ahora con recursos limitados.
	}%
}

% Please add the following required packages to your document preamble:
% \usepackage{graphicx}
% \usepackage[table,xcdraw]{xcolor}
% If you use beamer only pass "xcolor=table" option, i.e. \documentclass[xcolor=table]{beamer}
\begin{table}[H]
	\centering
	\caption{Resultados del ejercicio 7}
	\label{tab:ejercicio7}
	\resizebox{0.5\textwidth}{!}{%
		\begin{tabular}{|c|c|c|}
			\hline
			\rowcolor[HTML]{EFEFEF} 
			\textbf{ejercicio}                                     & \textbf{Nºacciones} & \textbf{Tiempo(s)} \\ \hline
			\cellcolor[HTML]{DAE8FC}\textbf{6 (sin restringir)}    & 61                  & 2.70               \\ \hline
			\cellcolor[HTML]{DAE8FC}\textbf{6 (restringido <= 28)} & 58                  & 298.56             \\ \hline
			\cellcolor[HTML]{DAE8FC}\textbf{6 (proven unsolvable)} & {\color[HTML]{FE0000} unsolvable} & $\infty$ \\ \hline
		\end{tabular}%
	}
\end{table}
Nº de acciones: 
\begin{itemize}
	\item Sin restringir: 26 de recolectar, 25 de navegar, 5 de reclutar, 3 de construir y 2 de asignar. No es óptimo.
	\item Restringido: 25 de recolectar, 23 de navegar, 5 de reclutar, 3 de construir y 2 de asignar. Es óptimo.
\end{itemize}

Tiempo: esta vez al restringir el tiempo aumenta por encima de los 2 segundos. Tarda casi 6 minutos en obtener una solución con menor número de pasos. Se prueba 'unsolvable' tras pasar 30 minutos esperando por debajo de 58 acciones.

%% Please add the following required packages to your document preamble:
% \usepackage{graphicx}
% \usepackage[table,xcdraw]{xcolor}
% If you use beamer only pass "xcolor=table" option, i.e. \documentclass[xcolor=table]{beamer}
\begin{table}[H]
	\centering
	\caption{Resultados de los ejercicios 1 al 7}
	\label{tab:resultados}
	\resizebox{0.6\textwidth}{!}{%
		\begin{tabular}{|
				>{\columncolor[HTML]{DAE8FC}}c |c|c|}
			\hline
			\cellcolor[HTML]{EFEFEF}\textbf{ejercicio} & \cellcolor[HTML]{EFEFEF}\textbf{Nºacciones} & \cellcolor[HTML]{EFEFEF}\textbf{Tiempo(s)} \\ \hline
			\textbf{1}                     & 5                                 & 0.00   \\ \hline
			\textbf{2}                     & 13                                & 0.00   \\ \hline
			\textbf{3}                     & 21                                & 0.07   \\ \hline
			\textbf{4}                     & 30                                & 0.01   \\ \hline
			\textbf{5}                     & 32                                & 0.04   \\ \hline
			\textbf{6 (sin restringir)}    & 32                                & 0.07   \\ \hline
			\textbf{6 (restringido <= 28)} & 28                                & 1.23   \\ \hline
			\textbf{6 (proven unsolvable)} & {\color[HTML]{FE0000} unsolvable} & 5.61   \\ \hline
			\textbf{7 (sin restringir)}    & 61                                & 2.70   \\ \hline
			\textbf{7 (restringido <= 28)} & 58                                & 298.56 \\ \hline
			\textbf{7 (proven unsolvable)} & {\color[HTML]{FE0000} unsolvable} & $\infty$\\ \hline
		\end{tabular}%
	}
\end{table}


\section{Pregunta \#1 }
\noindent\fbox{%
	\parbox{\textwidth}{\textbf{Enunciado:}
En las distintas llamadas a MetricFF necesarias para resolver el Ejercicio 6 (o cualquier otro en donde se busque optimizar de modo efectivo el tamaño del plan o algún otro criterio), ¿MetricFF suele tardar aproximadamente el mismo tiempo en todas ellas? ¿A qué cree que se debe este fenómeno? Razone su respuesta. 
	}%
}
\quad\\
Tanto en el ejercicio 6 como en el 7 observamos que al restringir más y más la solución, el planificador le cuesta más hallar una solución válida. Esto se debe a:
\begin{itemize}
	\item \textbf{Un espacio de búsqueda más reducido:} Al restringir la solución, se restringe el espacio de búsqueda y se reducen las posibilidades de encontrar una secuencia de acciones que cumpla con los objetivos y restricciones impuestas. Esto hace que sea más difícil para el planificador encontrar una solución factible dentro de ese espacio de búsqueda más restringido.
	
	\item \textbf{Una mayor complejidad del problema:} Al restringir la solución, se estén imponiendo restricciones que aumentan la complejidad del problema (aumenta el árbol de búsqueda). Por ejemplo, al agregar restricciones acciones o tiempo. Esto puede hacer que la búsqueda sea más complicada y necesite más tiempo para encontrar una solución factible.
\end{itemize}

\section{Pregunta \#2}
\noindent\fbox{%
	\parbox{\textwidth}{\textbf{Enunciado:}
En base a los tiempos de ejecución obtenidos, al comportamiento general observado en los distintos ejercicios, y teniendo en cuenta que el dominio de planificación planteado en esta práctica es de dificultad baja o moderada, ¿cuáles considera que pueden ser las principales limitaciones de la planificación automática en otros dominios? ¿Cómo considera que escalaría la resolución de los problemas si incorporásemos muchas más acciones, objetos y nodos del mapa?
	}%
}
\quad\\

La planificación automática tiene varias limitaciones. Algunas de las principales son: 
\begin{itemize}
	\item \textbf{Espacio de búsqueda y complejidad:} A medida que se incrementa el número de acciones, objetos y nodos en el mapa, el espacio de búsqueda aumenta exponencialmente. Esto hace que la planificación automática sea computacionalmente muy costosa.
	
	\item \textbf{Representación del conocimiento: } La planificación automática depende de una representación precisa del conocimiento del dominio y las reglas de acción. En el mundo real la representación es incompleta o imprecisa por lo que hay dificultades para generar planes efectivos. 
	
	\item \textbf{Incertidumbre y cambios en el entorno: } La planificación automática puede enfrentar dificultades cuando hay incertidumbre en el entorno o cuando el entorno cambia dinámicamente. La incertidumbre puede surgir de la falta de información completa sobre el estado actual del mundo o de la imposibilidad de predecir completamente las consecuencias de las acciones.
\end{itemize}

Como bien se menciona en la lista previa, una de las mayores limitaciones es la escalabilidad. Si incorporásemos muchas más acciones, objetos y nodos en el mapa llegaría un punto que la planificación se volvería inmanejable. Para poder escalar los problemas de planificación se necesitaría mejorar el algoritmo de búsqueda, la heurística o paralelizar la ejecución.

\end{document}
